\section{Related Work}
\subsection{Appraisal Modelling} \label{app_model}
% % MARSSI\cite{gebhard2018marssi}
% A group of researchers attempted to implement a model that simulates emotional appraisal with regulation processes with a social signal interpretation, called MARSSI (Model of Appraisal, Regulation, and Social Signal Interpretation) \cite{gebhard2018marssi}. They have implemented a model based on the OCC appraisal theory proposed by \citeauthor{ALMA2005Gebhard} \cite{ALMA2005Gebhard}. The conceptual difference between the OCC appraisal theory and the appraisal theory proposed by Scherer, as mentioned earlier, is that the OCC model focuses on the structure, in contrast, the other focuses on the sequential and hierarchical nature of appraisal processes \cite{clore2013psychological}. In addition to the appraisal model, regulations are also considered, which refer to the psychological phenomenon that one's emotions are suppressed or changed to fit into the current situation, which potentially influences situational appraisal information. The classified social signal interpretations, along with the features related to appraisal and regulation, are input into Dynamic Bayesian Networks (DBNs). This design allows theory-based modelling of the structure and the learning of temporal dynamics in interpreting social signals.

% Provide context of appraisal modelling
Appraisal in the moment can be seen as the dynamics of appraisal during an interaction as defined earlier. In terms of capturing the dynamics of the internal states of a person, previous attempts are limited to capturing the dynamics of emotion during interactions \cite{}. It is also questionable if emotional expressions, which sometimes are used interchangeably in the context of emotion inference, are truly reflecting the internal states of one. To our best knowledge, the present study is the first attempt to model the dynamics of appraisal with regard to the situational interdependence scale. 

% Appraisal modelling using LLM has been attempted. The performance might not be the best but it gives at least some indication of appraisal. 
There are several attempts to quantify the components of the appraisal using large language models (LLM) \cite{broekens2023fine, feng2023affect, tak2023gpt, yongsatianchot2023investigating, zhan-etal-2023-evaluating}. LLM has a huge potential in analysing textual content. This is realized by the emergence of Transformer \cite{vaswani2017attention}. This architecture has achieved great success in natural language processing, relying on self-attention mechanisms to process input data in parallel, which makes it possible to capture long-range dependencies in a sequence of data efficiently. 

% An example of modelling dimensions of emotional appraisal using LLM from texts.
When it comes to applying LLMs to appraisal modelling, one of the examples is research by \citeauthor{zhan-etal-2022-feel} \cite{zhan-etal-2023-evaluating}. They investigated the usage of LLMs to predict dimensions of emotional appraisal (CPM) with zero-shot learning using the CovidET dataset \cite{zhan-etal-2022-feel}, which consists of covid-related posts on Reddit. These posts are likely more closely related to the appraisal of remembered events, as it's improbable that individuals are posting on Reddit while the event is ongoing.  Similar approaches have been adopted, employing text descriptions of specific scenarios \cite{broekens2023fine, tak2023gpt, yongsatianchot2023investigating}. This research demonstrates the potential of LLMs in extracting dimensions of appraisal from textual data. However, to the best of our knowledge, there have been no attempts so far to explore other aspects of appraisal, such as situation perception or situational interdependence.

% The result was that the chatGPT, out of five GPTs they compared, achieved the highest result of Mean Absolute Error (MAE) being 1.694 and Spearman's correlation being 0.388, and F1 score of the detection of NA being 0.918.

% Their experiment tried to predict the Likert-scale predictions of 24 appraisal dimensions using different LLMs (ChatGPT, FLAN-T5-XXL\cite{google/flan-t5-xxl}, Alpaca\cite{alpaca}, Dolly-V2). Each model performed the prediction of certain appraisal dimensions on the Likert scale in 1-9 or not applicable (NA) if it does not exist in the situation. It is carried out using the CovidET dataset \cite{zhan-etal-2022-feel}, which consists of covid-related posts on Reddit. For the evaluation, the human-annotated labelling of those 24 dimensions of each post was used as the ground truth, and 

% An example of modelling emotion labels in a conversation setting. 
While much of the research on employing LLMs for appraisal modelling focuses on short texts or scripted dialogue scenarios in controlled environments or social media posts, there has been limited exploration specifically targeting appraisal within conversational settings. One notable study conducted by \citeauthor{feng2023affect} \cite{feng2023affect} addresses this gap by analyzing each utterance to identify emotions within a conversational context using a set of emotional labels. The study found that, compared to state-of-the-art supervised approaches, LLMs exhibited lower performance in zero-shot learning scenarios. However, performance notably improved with instruction-following demonstrations, indicating the effective utilization of prompts. While the research highlights that LLMs still have some way to go to match the performance of state-of-the-art supervised models, it underscores their greater generalizability to other natural language processing tasks and robustness to errors in automatic speech recognition. Overall, this study unveils the potential of employing LLMs within conversation settings to infer abstract phenomena, such as emotion, from conversational text data.

\subsection{Predicting Appraisal}
% An example of predicting appraisal as remembered from behaviour. However it does not capture time-variance 
Research by \citeauthor{recorgnizing2021Dudzik} \cite{recorgnizing2021Dudzik} investigated recognizing perceived situational interdependence in face-to-face negotiations by exploiting facial expressions, upper body behaviour and non-verbal vocal behaviour. They used the aforementioned SIS to measure situational interdependence and built a model based on the Ridge Classifier to analyse multivariate time series of those behavioural features. Their main discovery is that people's behaviour seems to be predictable of the perceptions of conflict of interest and power, while the conversation partner's behaviour is for conflict of interest, future independence and information certainty. Their research focuses on predicting appraisal as remembered, which is recorded via post-interaction reports, from behavioural features directly instead of interring appraisal. Also, this research did not take the temporal dynamics into account, instead, they constructed "aggregated" features from the behavioural features by using ROCKET (Random Convolutional Kernel Transformation) \cite{dempster2021minirocket}, which lacked the implications of the temporal dynamics of how each behavioural feature in the moment has significance to the situational interdependence.

% LSTM has been used for capturing dynamics of interactions and predicting emotions in affective computing field.
Since the appraisal in the moment is time-variant, preserving the temporal dynamics when predicting appraisal as remembered becomes crucial. This is why the Long Short-Term Memory network (LSTM) is commonly employed in the field of affective computing \cite{ong2019modeling}. The LSTM architecture has hidden internal states that function as "memory," retaining information from the past to be utilized in the future. This inherent capability of LSTMs to capture long dependencies over time aligns well with predicting appraisal as remembered. 

% Peak-end rule
In the field of psychology, there is a widely known phenomenon called the "Peak-end rule", which states that people tend to evaluate the overall experience of an event based on its peak and the end rather than based on the sum or weighted average of overall experience \cite{kahneman2000evaluation}. (definition of peak) This theory implies that the appraisal as remembered of interaction is hugely or solely based on the peak and the end of the appraisal in the moment. (ADD AN EXAMPLE APPLICATION OF PEAKEND RULE)

% The predictive model takes the time variant appraisal in the moment in terms of situational interdependence as its input and the predicted values of appraisal as remembered in terms of situational interdependence as its output. In order to preserve the time variant nature of appraisal for the prediction, Long term Short term Memory network (LSTM) is used. It has been used widely among other research in the field of affective computing \cite{ong2019modeling}. This architecture has hidden internal states where it works as "memory" and keep the information from the past to be used in the future. This nature of LSTM that is able to capture long dependency over time fits well with our task. 

% Also, the context of the conversations was that it was face-to-face and the goal was to negotiate with the partner, so it left room for how the finding would change for the conversations with different goals.