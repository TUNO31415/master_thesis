\section{Introduction} \label{intro}
% Understanding of human appraisal is useful for intelligent systems
One of the challenges in the field of intelligent systems is to understand how individuals subjectively appraise situations in interactions. Appraisal is defined as the process of evaluating situations based on personal belief, relevance, and significance \cite{scherer2005emotions}. This type of information is particularly beneficial for adaptive intelligence systems where they try to change their behaviours and responses such that they fit a user at a specific moment in time. A better understanding of users' perception of a situation is necessary to make the learning process more effective by identifying the specific pattern, phenomenon or social signals that can be utilized for inference of human perception. Traditionally, psychologists try to structurally uncover the process of how one perceives and evaluates a situation from different perspectives to understand different aspects of a situation, such as emotions \cite{scherer2005emotions, scherer2013nature}, situation perception \cite{rauthmann2014situational}, or situational interdependence \cite{gerpott2017howdopeople}.

% back up with literature, write stuff down, explainability, definition of "appraisal"

% One example is situation perception and situational interdependence 
As an example of different aspects of appraisal, situation perception is a type of cognitive process that interprets and understands situations subjectively based on an individual's experiences, beliefs, and expectations \cite{tekoppele2023we}. The expressed emotions and behaviour can be indicative of how one perceives a situation \cite{hess2020bidirectional, horstmann2019situational}. When it comes to different aspects of evaluation, as one of the facets of situation perception, there is situational interdependence, which particularly highlights the perceived interdependence in a certain situation. The situational Interdependence Scale (SIS) proposed by \citeauthor{gerpott2017howdopeople} \cite{gerpott2017howdopeople} provides a way to quantify subjective (or perceived) interdependence based on the five dimensions (mutual dependence, conflict of interest, future independence, information certainty, and power) that articulate individuals' perceptions of their interdependence. 

\begin{table}
    \centering
    \begin{tabular}{|c|l|} \hline
        Dimension & Description\\\hline
       Mutual Dependence  & \\
       Conflict of Interest  & \\
       Future Independence  & \\
       Information Certainty  & \\
       Power  & \\ \hline
    \end{tabular}
    \caption{Dimensions of situational interdependence \cite{gerpott2017howdopeople}}
    \label{tab:sis_dimensions}
\end{table}

% Another example is Appraisal process of emotion
Another specific example of different aspects of cognitive appraisal, the component process model (CPM) of appraisal is inherited from the appraisal theory of emotion, in which specific configurations of those dimensions in combination define emotional labels, such as sadness, happiness and more \cite{sander2005systems, scherer2013nature}. The components consist of relevance, implications/consequences, coping potential, and norm compatibility. Individual differences in perception make its process subjective, which leads to diverse outcomes when evaluating the same stimulus event. This variation occurs across different individuals and within the same individual over time, influenced by changes in perception, and the situation \cite{dudzik2023valid}. The temporal distance between the moment of appraisal and the moment of the stimulus event also has an impact on its evaluation \cite{trope2003temporal}. The outcome of the appraisal can be inferred from observable cues, thus one can estimate the internal cognitive states of a person by exploiting facial expressions \cite{kaiser2001facial}, verbal contents \cite{}, verbal cues \cite{Lotfian2019building}, or physiological signals \cite{}. 

% Appraisal in the moment vs appraisal as remembered 

\begin{table}
    \centering
    \begin{tabular}{|c|l|} \hline
        Dimension & Description\\\hline
        Relevance  & \\
       Implications/Consequences  & \\
       Coping potential  & \\
       Norm compatibility  & \\ \hline
    \end{tabular}
    \caption{Components of component process model from appraisal theory of emotion  \cite{scherer2013nature}}
    \label{tab:emo_dimensions}
\end{table}

% Explain situation perception
% As you can see, the appraisal process and situation perception function as frameworks for the systematical assessment of the situation and the conversation partner in the field of cognitive science. These frameworks help to understand how these evaluations are systematically conducted. 
% The appraisal process and situation perception serve as conceptual frameworks within the domain of cognitive science, facilitating the systematic assessment of both the situation and the conversational partner. These frameworks contribute to a comprehensive understanding of the systematic procedures involved in conducting evaluations. 

% The problems with modelling appraisal in a situation. One being the complexity of the cognitive process.
Having explained what appraisal is and specific examples of different constructs, modelling appraisal in a situation can be inherently challenging. One major obstacle lies in the difficulty of precisely determining the extent to which each component of the appraisal process is triggered from observable signals, regardless of the specific constructs of appraisal being considered. This is due to the complexity of the cognitive system, which complicates both the data collection and annotation processes \cite{sander2005systems}.

% From multimodal sensory data -> ML -> DB
% Appraisal in the moment is not labelled. 

% Another problem being hard to find the optimized sweet spots of "thin slice". 
Another problem is that the appraisal process is known to be temporally dynamic and sequential, presenting challenges in determining an optimal frequency for data collection \cite{tekoppele2023we}. The temporal distances between the appraising stimulus event and the reporting moment can significantly influence participants' appraisal processing \cite{dudzik2023valid}. When observing conversations to extract indications of appraisal from observable cues for modelling, it is important to appropriately slice interactions \cite{murphy2021capturing}. Therefore, ensuring conceptual validity and computational feasibility, as well as understanding the assumptions and limitations, are imperative.

% The Last problem is that current research did not really focus on distinguishing the appraisal in a moment and as remembered. 
Lastly, while it may seem intuitive to assume a direct correlation between the appraisal in the moment and the appraisal as remembered, this connection remains ambiguous. One primary reason for this ambiguity is the conceptual disparity between these two forms of appraisal. The appraisal in the moment deals with the ongoing situation and stimuli, whereas the latter evaluates the remembered interactions. It is necessary to distinguish between these two because the temporal gap between the stimulus and the moment of appraisal influences the assessment\cite{trope2003temporal}. However, current research has not extensively delved into this differentiation; instead, it has primarily focused on the appraisal as remembered. For instance, researchers often utilize post-interaction questionnaires, treating them as an appraisal of the situation.

% and how different contextual settings of interactions influence its articulation.

Acknowledging these gaps in research and limitations, this research is going to investigate the following research questions to uncover the relationship between the appraisal in the moment and the appraisal of remembered interactions. 

\begin{enumerate}
    \item How does appraisal of a situation in the moment shape appraisal of the same situation as remembered?
    \item How does that change in different settings of conversations?
    % \item How does the appraisal process shape an individual’s summary evaluation of situational interdependence after an interaction?
    % \item How do articulations of an individual’s summary evaluation of situational interdependence differ over different contextual differences in conversations?
\end{enumerate}

% explain that 1) model appraisal in the moment using observable signals, and 2) it is used to predict appraisal as remembered so that it is possible to find the correlation. 
To address these questions, two datasets that contain video recordings of dyadic conversations and a post-conversation questionnaire for evaluating the interaction are used, essentially representing the appraisal as remembered. With these datasets, two modelling steps are carried out. Firstly, appraisal in the moment during a conversation is modelled as a time-series signal. The conversation is segmented appropriately to maintain conceptual coherence. At each segmented time window, the appraisal in that specific moment is estimated by leveraging observable signals (video recordings) known to indicate appraisal from literature. Subsequently, this time-series data, representing the evolving appraisal in the moment throughout the conversation, is fed into a machine learning model. This model aims to predict and estimate how one appraises the remembered conversation. The resulting predictions are then analyzed to understand the correlation between the appraisal in the moment and the appraisal as remembered. The details of the datasets and methodologies are discussed later in this paper.

% Hypothesis 1: certain dimensions of appraisal in the moment have unique impacts on certain dimensions of appraisal as remembered
One of the hypotheses of the first research question is that there exist certain dimensions of an individual's appraisal that are closely linked to indicating their appraisal as remembered. \citeauthor{gerpott2017howdopeople} \cite{gerpott2017howdopeople} investigated how a participant felt during an interaction, using four emotion labels (anger, disgust, happiness and sadness) in a situation and tried to find a correlation with the situational interdependence, which represent appraisal as remembered. The report of the emotion labels from participants can be seen as the appraisal in the moment in terms of emotion labels, as they are asked to recall how they felt during interactions instead of reevaluating the interaction. The results support that each dimension correlates with specific emotions. For example, more participants felt sad in a situation where they perceived the situation with a higher conflict. Thus, we can hypothesize the claim that certain dimensions of an individual's appraisal in the moment can be indicative of certain dimensions of appraisal as remembered. 

% Hypothesis2: The peak and the end of appraisal in the moment have a huge impact on appraisal as remembered
In terms of the time variance, the peak-end rule states that people tend to evaluate the overall experience of an event based on its peak and the end rather than based on the sum or weighted average of overall experience \cite{kahneman2000evaluation}. This allows us to hypothesise that the peak and the end of one's appraisal in the moment have more impacts on one's appraisal as remembered compared to other parts of the conversation. 

% Hypothesis of RQ2
% Contextual differences make a difference in how appraisal in the moment can indicate the appraisal as remembered. Because the contextual differences affect some dimensions of situational interdependence. 

% Regarding the hypothesis of the second research question, the relationship between appraisal in the moment and as remembered is expected to vary depending on contextual differences. This is based on the fact that the contextual situation influences the appraisal in general. It is found in some research that contextual settings serve as determinants of perceived situational interdependence \cite{hess2020bidirectional, recorgnizing2021Dudzik}, which this finding can be interpreted as the appraisal of remembered interactions changed . Also, emotional stimuli are also known to be evaluated differently under different situational dynamics \cite{hess2020bidirectional}, which implies that appraisal in the moment would change. 